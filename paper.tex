%% BioMed_Central_Tex_Template_v1.05
%%                                      %
%  bmc_article.tex            ver: 1.05 %
%                                       %


%%%%%%%%%%%%%%%%%%%%%%%%%%%%%%%%%%%%%%%%%
%%                                     %%
%%  LaTeX template for BioMed Central  %%
%%     journal article submissions     %%
%%                                     %%
%%         <27 January 2006>           %%
%%                                     %%
%%                                     %%
%% Uses:                               %%
%% cite.sty, url.sty, bmc_article.cls  %%
%% ifthen.sty. multicol.sty		      %%
%%									   %%
%%                                     %%
%%%%%%%%%%%%%%%%%%%%%%%%%%%%%%%%%%%%%%%%%


%%%%%%%%%%%%%%%%%%%%%%%%%%%%%%%%%%%%%%%%%%%%%%%%%%%%%%%%%%%%%%%%%%%%%
%%                                                                 %%	
%% For instructions on how to fill out this Tex template           %%
%% document please refer to Readme.pdf and the instructions for    %%
%% authors page on the biomed central website                      %%
%% http://www.biomedcentral.com/info/authors/                      %%
%%                                                                 %%
%% Please do not use \input{...} to include other tex files.       %%
%% Submit your LaTeX manuscript as one .tex document.              %%
%%                                                                 %%
%% All additional figures and files should be attached             %%
%% separately and not embedded in the \TeX\ document itself.       %%
%%                                                                 %%
%% BioMed Central currently use the MikTex distribution of         %%
%% TeX for Windows) of TeX and LaTeX.  This is available from      %%
%% http://www.miktex.org                                           %%
%%                                                                 %%
%%%%%%%%%%%%%%%%%%%%%%%%%%%%%%%%%%%%%%%%%%%%%%%%%%%%%%%%%%%%%%%%%%%%%


\NeedsTeXFormat{LaTeX2e}[1995/12/01]
\documentclass[10pt]{bmc_article}    



% Load packages
\usepackage{cite} % Make references as [1-4], not [1,2,3,4]
\usepackage{url}  % Formatting web addresses  
\usepackage{ifthen}  % Conditional 
\usepackage{multicol}   %Columns
\usepackage[utf8]{inputenc} %unicode support
%\usepackage[applemac]{inputenc} %applemac support if unicode package fails
%\usepackage[latin1]{inputenc} %UNIX support if unicode package fails
\urlstyle{rm}
 
 
%%%%%%%%%%%%%%%%%%%%%%%%%%%%%%%%%%%%%%%%%%%%%%%%%	
%%                                             %%
%%  If you wish to display your graphics for   %%
%%  your own use using includegraphic or       %%
%%  includegraphics, then comment out the      %%
%%  following two lines of code.               %%   
%%  NB: These line *must* be included when     %%
%%  submitting to BMC.                         %% 
%%  All figure files must be submitted as      %%
%%  separate graphics through the BMC          %%
%%  submission process, not included in the    %% 
%%  submitted article.                         %% 
%%                                             %%
%%%%%%%%%%%%%%%%%%%%%%%%%%%%%%%%%%%%%%%%%%%%%%%%%                     


\def\includegraphic{}
\def\includegraphics{}



\setlength{\topmargin}{0.0cm}
\setlength{\textheight}{21.5cm}
\setlength{\oddsidemargin}{0cm} 
\setlength{\textwidth}{16.5cm}
\setlength{\columnsep}{0.6cm}

\newboolean{publ}

%%%%%%%%%%%%%%%%%%%%%%%%%%%%%%%%%%%%%%%%%%%%%%%%%%
%%                                              %%
%% You may change the following style settings  %%
%% Should you wish to format your article       %%
%% in a publication style for printing out and  %%
%% sharing with colleagues, but ensure that     %%
%% before submitting to BMC that the style is   %%
%% returned to the Review style setting.        %%
%%                                              %%
%%%%%%%%%%%%%%%%%%%%%%%%%%%%%%%%%%%%%%%%%%%%%%%%%%
 

%Review style settings
\newenvironment{bmcformat}{\begin{raggedright}\baselineskip20pt\sloppy\setboolean{publ}{false}}{\end{raggedright}\baselineskip20pt\sloppy}

%Publication style settings
%\newenvironment{bmcformat}{\fussy\setboolean{publ}{true}}{\fussy}



% Begin ...
\begin{document}
\begin{bmcformat}


%%%%%%%%%%%%%%%%%%%%%%%%%%%%%%%%%%%%%%%%%%%%%%
%%                                          %%
%% Enter the title of your article here     %%
%%                                          %%
%%%%%%%%%%%%%%%%%%%%%%%%%%%%%%%%%%%%%%%%%%%%%%

\title{Open Data, Open Source and Open Standards in chemistry - The Blue Obelisk five years on}
 
%%%%%%%%%%%%%%%%%%%%%%%%%%%%%%%%%%%%%%%%%%%%%%
%%                                          %%
%% Enter the authors here                   %%
%%                                          %%
%% Ensure \and is entered between all but   %%
%% the last two authors. This will be       %%
%% replaced by a comma in the final article %%
%%                                          %%
%% Ensure there are no trailing spaces at   %% 
%% the ends of the lines                    %%     	
%%                                          %%
%%%%%%%%%%%%%%%%%%%%%%%%%%%%%%%%%%%%%%%%%%%%%%


\author{Charles A Darwin\correspondingauthor$^{1,2}$%
       \email{Charles A Darwin\correspondingauthor - charles@londonzoo.co.uk}%
      \and 
         Egon L Willighagen$^3$%
         \email{Egon L Willighagen - egon.willighagen@ki.se}
     \and 
         Rajarshi Guha$^4$%
         \email{Rajarshi Guha - guhar@mail.nih.gov}
	 \and 
	     Christoph Steinbeck$^6$%
	         \email{Christoph Steinbeck - steinbeck@ebi.ac.uk}
     \and 
         Noel M O'Boyle$^5$%
         \email{Noel M O'Boyle - n.oboyle@ucc.ie}
     \and 
         Ola Spjuth$^9$%
         \email{Noel M O'Boyle - n.oboyle@ucc.ie}
     \and 
         Jonathan Alvarsson$^9$%
         \email{Jonathan Alvarsson - jonathan.alvarsson@farmbio.uu.se}
     \and 
         Peter Murray-Rust$^7$%
         \email{Peter Murray-Rust - pm286@cam.ac.uk}
     \and 
         Daniel M Lowe$^7$%
         \email{Daniel Lowe - dl387@cam.ac.uk}
     \and
         Robert Hanson$^8$%
         \email{Robert Hanson - hansonr@stolaf.edu}
      \and
         Dmitry Pavlov$^8$%
         \email{Dmitry Pavlov - dpavlov@ggasoftware.com}
      \and
         Igor V Filippov\correspondingauthor$^{10}$%
         \email{Igor V Filippov\correspondingauthor - igorf@helix.nih.gov}
      \and
         Jane E Doe\correspondingauthor$^2$%
         \email{Jane E Doe\correspondingauthor - jane.e.doe@cambridge.co.uk}
      }
      

%%%%%%%%%%%%%%%%%%%%%%%%%%%%%%%%%%%%%%%%%%%%%%
%%                                          %%
%% Enter the authors' addresses here        %%
%%                                          %%
%%%%%%%%%%%%%%%%%%%%%%%%%%%%%%%%%%%%%%%%%%%%%%

\address{%
    \iid(1)Life Sciences Department, Kings College London, Cornwall House,%
        Waterloo Road, London, UK\\
    \iid(2)Department of Zoology, Cambridge, Waterloo Road, London, UK\\
    \iid(3)Division of Molecular Toxicology, Institute of Environmental Medicine, %
        Nobels vaeg 13, Karolinska Institutet, 171 77 Stockholm, Sweden\\
    \iid(5)Analytical and Biological Chemistry Research Facility, Cavanagh Pharmacy Building, University College Cork, College Road, Cork, Co. Cork, Ireland\\
    \iid(6)Cheminformatics and Metabolism Team, European Bioinformatics Institute (EBI), Wellcome Trust Genome Campus, Hinxton, Cambridge, UK\\
    \iid(7)Unilever Centre for Molecular Science Informatics, Department of Chemistry, University of Cambridge, Lensfield Road, CB2 1EW, UK\\
    \iid(8)GGA Software Services LLC, 41 Nab. Chernoi rechki 194342, Saint Petersburg, Russia\\
    \iid(9)Department of Pharmaceutical Biosciences, Uppsala University, Box 591, 751 24 Uppsala, Sweden \\
    \iid(10)NIH, Somewhere in the US, USA
}%

\maketitle

%%%%%%%%%%%%%%%%%%%%%%%%%%%%%%%%%%%%%%%%%%%%%%
%%                                          %%
%% The Abstract begins here                 %%
%%                                          %%
%% The Section headings here are those for  %%
%% a Research article submitted to a        %%
%% BMC-Series journal.                      %%  
%%                                          %%
%% If your article is not of this type,     %%
%% then refer to the Instructions for       %%
%% authors on http://www.biomedcentral.com  %%
%% and change the section headings          %%
%% accordingly.                             %%   
%%                                          %%
%%%%%%%%%%%%%%%%%%%%%%%%%%%%%%%%%%%%%%%%%%%%%%


\begin{abstract}
        % Do not use inserted blank lines (ie \\) until main body of text.
        \paragraph*{Background:} The Blue Obelisk movement was established in 2005 as a 
		response to the lack of open data, open standards and
		open source (ODOSOS) in chemistry. While other scientific disciplines
		such as physics, biology and astronomy (to name a few) were embracing
		new ways of doing science and reaping the benefits of community
		efforts, there was little if any innovation in the field of chemistry
		and scientific progress was actively hampered by the lack of access to
		data and tools. 
      
        \paragraph*{Results:} This
		contribution looks back on the past 5 years and surveys progress and
		remaining challenges in the areas of Open Source, Open Data and Open
		Standards in chemistry.

        \paragraph*{Conclusions:} Here we show that the Blue Obelisk has been very successful
		in bringing together researchers and developers with common interests
		in ODOSOS, leading to development of many useful resources freely
		available to the chemistry community. But how best to engage with the
		wider chemistry community outside of the Blue Obelisk remains an open
		question.
\end{abstract}



\ifthenelse{\boolean{publ}}{\begin{multicols}{2}}{}




%%%%%%%%%%%%%%%%%%%%%%%%%%%%%%%%%%%%%%%%%%%%%%
%%                                          %%
%% The Main Body begins here                %%
%%                                          %%
%% The Section headings here are those for  %%
%% a Research article submitted to a        %%
%% BMC-Series journal.                      %%  
%%                                          %%
%% If your article is not of this type,     %%
%% then refer to the instructions for       %%
%% authors on:                              %%
%% http://www.biomedcentral.com/info/authors%%
%% and change the section headings          %%
%% accordingly.                             %% 
%%                                          %%
%% See the Results and Discussion section   %%
%% for details on how to create sub-sections%%
%%                                          %%
%% use \cite{...} to cite references        %%
%%  \cite{koon} and                         %%
%%  \cite{oreg,khar,zvai,xjon,schn,pond}    %%
%%  \nocite{smith,marg,hunn,advi,koha,mouse}%%
%%                                          %%
%%%%%%%%%%%%%%%%%%%%%%%%%%%%%%%%%%%%%%%%%%%%%%




%%%%%%%%%%%%%%%%
%% Background %%
%%
\section*{Background}
The Blue Obelisk movement was established in 2005 at the
229\textsuperscript{th} National Meeting of the American Chemistry
Society as a response to the lack of open data, open standards and
open source (ODOSOS) in chemistry. While other scientific disciplines
such as physics, biology and astronomy (to name a few) were embracing
new ways of doing science and reaping the benefits of community
efforts, there was little if any innovation in the field of chemistry
and scientific progress was actively hampered by the lack of access to
data and tools.
Since 2005 it has become evident that a good amount of development in open 
chemical information and information is driven by the demands of neighbouring
scientific fields. In many areas in biology, for example, the importance of 
small molecules and their interactions and reactions in biological systems
has been realised. In fact, one of the first free and open databases and ontologies 
of small molecules was created as a resource about chemical structure and nomenclature
by biologists \cite{DeMatos:2009p3839}.

The formation of the Blue Obelisk group is somewhat unusual in that it
is not a funded network, nor does it follow the industry consortium
model. Rather it is a grassroots organisation, catalysed by an initial
core of interested scientists, but with membership open to all who
share one or more of the goals of the group:
\begin{itemize}
\item Open Source in Chemistry. One can use other people's code without further
permission, including changing it for one's own use and distributing
it again.
\item Open Standards in Chemistry. One can find visible community mechanisms for
protocols and communicating information. The mechanisms for creating
and maintaining these standards cover a wide spectrum of human
organisations, including various degrees of consent.
\item Open Data in Chemistry. One can obtain all data in the public domain when
wanted and reuse it for whatever purpose.
\end{itemize}

Note that while some may advocate also for Open Access to
publications, the goals focus more on the availability of code (to
reproduce results), standards (to exchange data), and the scientific
data itself. All three of these goals stem from the basics of the
scientific method for data sharing and reproducibilty.

The Blue Obelisk was first described in the CDK News \cite{CDKNewsBO} and
later as a formal paper by Guha et al.\cite{Guha2006} in
2006. This
contribution looks back on the past 5 years and surveys progress and
remaining challenges in the areas of Open Source, Open Data and Open
Standards in chemistry.

%%%%%%%%%%%%%%%%%%%%%%%%%%%%
%% Main section %%
%%
\section*{Open Source}
  \subsection*{Progress}

\subsubsection*{Cheminformatics toolkits}

Open Source toolkits for cheminformatics have now existed for nearly
ten years. During this period, some toolkits were developed from
scratch in academia, whereas others were made Open Source by releasing in-house
codebases under liberal licenses. When the Blue Obelisk was
established five years ago, the primary toolkits under active development
were the Chemistry Development Kit (CDK)
\cite{Steinbeck2003, Steinbeck2006}, Open Babel \cite{WebOpenBabel},
and JOELib \cite{WebJOELib}. Of these, both the CDK and Open Babel
continue to be actively developed. [Insert main focus of work on CDK
in last 5 years] Since 2006, major new features of Open Babel include 3D structure
generation and 2D structure-diagram generation, UFF and MMFF94
forcefields, and significantly expanded support for computational
chemistry calculations. In addition, a major focus of Open Babel development
has been to fix problems in the areas of stereochemistry, kekulisation
and canonicalisation.

Some project have also been adopted by other fields, including biology
and crystallography. A good example
here is Jmol, which has, in the past five years, gone through two major revisions
as part of an accelerated development focus. Starting with Jmol 10.2, released in
April of 2006, Jmol 11.0 added over 100 new features and was released in March of
2007. Three years later, Jmol 12.0 followed with over 700 new features.
Based on the outstanding work of Michael (Miguel) Howard and others, this development
path took Jmol from a ``Rasmol/Chime replacement'' to a fully fledged molecular
visualisation package, including full support for crystallography~\cite{Hanson2010},
efficient delivery of surface data using fast solvent-excluded surface
generation and the highly compressed JVXL format [http://chemapps.stolaf.edu/jmol/docs/misc/JVXL-format.pdf],
 display of molecular orbitals from standard basis set/coefficient data,
the inclusion of dynamic minimisation using the UFF force field, and
a full implementation of Daylight SMILES and SMART, with extensions to
conformational and biomolecular substructure searching in the form of Jmol
BioSMARTS.[http://jmol.svn.sourceforge.net/viewvc/jmol/trunk/Jmol/src/org/jmol/smiles/package.html]
Behind the scenes, Jmol has been essentially completely rewritten. A full
mathematical expression evaluator was added, and the scripting language
was folded into a versatile JavaScript-like command language, including a
complete set of program flow commands, several new data types, and
user-defined variables and functions. An exciting non-chemical application
of Jmol has been its incorporation into the open source Sage
Mathematics Project [http://www.sagemath.org/], where Jmol is being used
to deliver three-dimensional depiction of mathematical functions. 

Two new Open Source cheminformatics toolkits have appeared since the
original paper. In 2006 Rational Discovery, a cheminformatics service
company (since closed down), released RDKit \cite{WebRDKit} under the
BSD License. RDKit continues to be actively developed and includes
code donated by Novartis.

More recently, GGA Software Services (a contract programming company)
released the Indigo toolkit \cite{WebIndigo} and associated software
in 2009 under the GPL. Indigo is written from scratch in C++ and has
high-level wrappers for using it in C, Java, Python, or .NET
environment. Like other toolkits, Indigo provides support for
tetrahedral and cis-trans stereochemistry, 2D coordinate generation,
exact/substructure/SMARTS matching, fingerprint generation (all these
both for molecules and reactions), and canonical SMILES computation.
It also provides some less common functionality, like matching
tautomers and resonance substructures, enumeration of subgraphs,
finding maximum common substructure of $N$ input structures, and
enumerating reaction products. Indigo also has its rendering plugin,
which is used in a lot of commercial and open-source projects as the
main 2D chemical rendering engine.

While desktop software has composed the majority of scientific tools since the computer was introduced, the internet continues to change how applications and content are distributed and presented. As such, there is a growing need to present chemical and scientific information in a web-based medium. Recently, a new version of the HTML specification, HTML5 \cite{html5}, defines a well-developed framework for creating native web applications in JavaScript. This native framework provides a significant benefit to scientists as the web is an open and free medium to distribute scientific knowledge, ideas and education. It also poses significant new problems, as the cost associated with learning and developing tools for new technologies is not trivial. As such, it is an important goal for the Blue Obelisk to provide such tools and encourage the community to utilise these new technologies to improve products and education. A specific example of this initiative is the ChemDoodle Web Components library \cite{ChemDoodleWeb}, produced by iChemLabs. The ChemDoodle Web Components library is open source under the GPL v3 license with a liberal HTML exception. ChemDoodle Web Components allow the scientist to present publication quality 2D and 3D graphics and animations for chemical structures, reactions and spectra. Beyond graphics, this tool provides a framework for user interaction to create dynamic applications through web browsers, desktop platforms and mobile devices such as the iPhone, iPad and Android devices. It is now easier than ever for scientists to create web-based scientific content, without having to master complicated technologies such as Java applets. iChemLabs is dedicated to funding, developing and supporting the library to ensure that the next generation of scientific applications is easily achievable by academia, government and industry. This goal helps to make sure that the cost of education decreases while using the web to further spread science.

\subsubsection*{Second-generation tools}

Although feature-rich and robust cheminformatics toolkits are useful
in and of themselves, they can also be seen as providing a base layer
on which additional tools and applications can be built. This is one
of the reasons that cheminformatics toolkits are so important to the
open source `ecosystem'; their availability lowers the barrier for the
development of a `second generation' of chemistry software that no
longer needs to concern itself with the low-level details of
manipulating chemical structures, and can focus on providing
additional functionality and ease-of-use.

Bioclipse~\cite{Spjuth:2007fk} (v2.4 released in Aug 2010) and Avogadro
\cite{WebAvogadro} (v1.0 in Oct 2009) are two examples of such software, based
on the CDK and Open Babel, respectively. Bioclipse is an award-winning
molecular workbench for life sciences with a dynamic plugin architecture, and
which wraps cheminformatics functionality behind user-friendly interfaces and
graphical editors. All functionality in Bioclipse is completely scriptable
using an integrated high-level language called Bioclipse Scripting Language
(BSL), which can be used for scripted research or to automate complex
tasks~\cite{Bioclipse2}. Bioclipse in turn acts as base for other software
such as Brunn~\cite{Brunn}, which is a laboratory information system for
microplate based high-throughput screening. Brunn provides a grahical interface
for handling different plate layouts and dilution series. It can automaticly
generate dose response curves and calculate IC$_{50}$-values. In this manner
open source projects empower other open source projects.

Avogadro is a 3D molecular editor and viewer aimed at preparing
and analysing computational chemistry calculations. An interesting
aspect of both Avogadro and Bioclipse is that they share some developers
with the underlying toolkits and this has driven the development of new
features in the CDK and Open Babel.

CDK-Taverna \cite{Kuhn:2010p4001} as an open-source cheminformatics workflow solution.
Commonly used in this context are workflow engines for cheminformatics, where numerous recurring tasks can be automated, including tasks for chemical data filtering, transformation, curation and migration workflows, chemical documentation and information retrieval related workflows (structures, reactions, pharmacophores, object relational data etc.) or data analysis workflows (statistics and clustering/ machine learning for QSAR, diversity analysis etc. 
CDK-Taverna has matured to become a freely available and increasingly powerful tool for the biosciences. It was recently ported to 
Taverna version 2.x.
The combination of CDK-Taverna workers with the multitude of workers already published on myexperiment.org by the large and active user community of Taverna enables scientists to quickly build workflows to process a diverse set of data as typically found in today's systems biology scenarios.

Some other recent projects that build on Blue Obelisk software include
AMBIT (a GUI that facilitates registration of chemicals for the REACH
EU directive on toxicity, based on the CDK), etc.

\subsubsection*{The business end}

Open Source provides a unique opportunity for commercial organisations to work with the community. As opposed to traditional business models which focus on maximising revenue through monetisation of all aspects of software, working with the community allows a business to maximise profits through greater community adoption and popularity. This is an ideal solution for some commercial organisations because only they can afford the significant investment in funding and time that is required to develop a successful open source project. Regardless, there are always challenges to growing a project, and there are specific issues that complicate making a business out of open source software in the scientific industry. A goal of the Blue Obelisk is to reduce these challenges and bring commercial organisations and the community together to benefit science.

Open source products attract interest because they are easily modifiable, and they can usually be used freely under a standard open source license. With proper advertising, a company can grow a user base more quickly than with a proprietary product, which requires significantly more marketing resources. Additionally, an open source product may integrate with other products that are governed by compatible open source licensing, allowing for faster development. Adopters benefit from a quickly established community of users, that not only know how to use that software, but may also have experience developing it. These benefits allow a company to create a successful product with significantly reduced cost and effort.

Successful open projects require the dedication of a number of individuals to ensure that the product is continually developed, supported and funded. In many non-commercial cases, an author will produce an open project as a hobby, and stop work on the project before the development reaches a certain quality or community interest reaches a critical mass. This is because the hobbyist eventually needs to focus on income or he/she has a change of priorities. On the other hand, commercial organisations have the means to devote resources to a project to ensure that it achieves an appropriate level of quality and adoption. In return, that commercial organisation gains recognition from that product and is the first choice for commercial support, consulting and custom development. This is the primary goal of commercial open source, as selling licenses for an open product is very unlikely to be feasible.

In the scientific industry, it is incredibly difficult to start successful open projects. This is because the expertise needed to contribute to such a project requires a higher degree of education and the population of end users is much smaller than other industries, such as entertainment. Because of this, there is an increased barrier to starting open projects in the scientific industry. The Blue Obelisk attempts to solve this issue, by uniting those interested in open source and by helping to increase awareness for new open projects. Commercial organisations can take advantage of this goal, as open products will provide a faster return on investment, and the community benefits from open software and new job opportunities.

For instance, iChemLabs distributes several scientific software solutions under various open and proprietary licensing. The license is chosen to best ensure the success of the project. iChemLabs released the ChemDoodle Web Components library under the GPL v3 license with a liberal HTML exception. iChemLabs is dedicated to developing, funding and supporting the library so that the scientific community can immediately utilise the maturing HTML5 technologies and quickly make web and mobile interfaces for scientific content. Other open source developers can benefit from this library freely and expect continued development, while commercial organisations that use the library under an appropriate license (GPL for open projects or purchasing a proprietary license for proprietary projects) will benefit from an already developed market. Of course, no one will need to waste time rewriting functionality that is already provided with quality. Several open projects have already incorporated the ChemDoodle Web Components, such as iBabel by Chris Swain\cite{iBabel}, ChemSpotlight by Geoffrey Hutchison\cite{chemspotlight} and the RSC ChemSpider\cite{chemspider_chemdoodle}.

An example of such an interaction was the donation by eMolecules, Inc.
to Open Babel of code for the canonicalisation of molecules and
fragments (Nov 2006). eMolecules is an online vendor of chemicals that
uses Open Babel under-the-hood to manage its compound collections.

Another more recent example occurred in July 2010 when Silicos, a
Belgian company that provides services
in the area of cheminformatics, released several command line
applications based on Open Babel as well as donating code to the
project. For example, the Pharao
tool released by them is a comprehensive solution for pharmacophore
searching that provides extensive support for a variety of
pharmacophore searches. Other tools released by Silicos include Sieve for
filtering by molecular property, Stripper for removing core scaffold
structures from a molecule set, and Piramid for molecular alignment
using shape determined by the Gaussian volumes as a descriptor.

Another example of Open Source
tools originating from commercial groups is the ChemCraft tool from
Molecular Networks GmBH, that does XXX.

Rather than releasing new tools, an alternative approach to provide an
interface to existing tools. hBar Solutions has developed an online
portal, hBar Lab, for managing and performing computational chemistry
calculations in the cloud. To do so it leverages two Blue Obelisk
projects, Jmol and Open Babel, as well as the open source quantum
mechanics packages MPQC.

\subsubsection*{Converting chemical names and images to structures}

The majority of chemical information is not stored in machine-readable
formats, but rather as chemical names or depictions. The OSRA and OPSIN
projects focus on extracting chemical information from these sources.
Such software plays a particularly important role for datamining the
chemical literature, including patents and theses.

Optical Structure Recognition Application (OSRA) \cite{WebOSRA} was started
in early 2007 with the goal to create first free and open source
tool for extraction and conversion of molecular images into SMILES and
SD files. From the very beginning the underlying philosophy was to integrate
existing open source libraries and to avoid ``reinventing the wheel''
wherever possible. OSRA relies on a variety of open source components:
Open Babel for chemical format
conversion and molecular property calculations, GraphicsMagick for image
manipulation, Potrace for vectorization, GOCR and OCRAD for optical
character recognition. The growing importance of image
recognition technology can be seen in the fact that 
only a few years ago there was only one widely available software
package for chemical structure recognition -  CLiDE (commercially
developed at Keymodule, Ltd), but today there are as many as seven
available programs.

The Blue Obelisk project OPSIN (Open Parser for Systematic IUPAC
Nomenclature)~\cite{lowe_chemical_2011} focuses instead on interpreting chemical names.
The chemical name is the oldest form of communication used to
describe chemicals, predating
even the knowledge of the atomic structure of compounds.
Chemical names are abundant in the scientific
literature and encode valuable structural information.
Through successive books of
recommendations~\cite{iupac_nomenclature_1979, iupac_guide_1993},
IUPAC have tried to codify and to an extent standardise naming practices.
OPSIN aims to make this abundance of
chemical names machine readable by translating them to SMILES, CML or
InChI. The program is based around the use of a regular grammar to
guide tokenisation and parsing of chemical names, followed by
step-wise application of nomenclature rules. OPSIN is able to offer
fast and precise conversions for the majority of names using IUPAC
organic nomenclature, and is available as a web service, Java
library and standalone application for maximum interoperability.

  \subsubsection*{Collaboration and interoperability}

One of the effects of the Blue Obelisk has been to bring developers
together from different Open Source chemistry projects so that they
look for opportunities to collaborate rather than compete, and to
leverage work done by other projects to avoid duplication of effort.
As an example of this, when in March 2008 the Jmol development team
were looking to add support for energy minimisation, rather than
implement a forcefield from scratch they ported the UFF forcefield
implementation from Open Babel to Jmol. This code has allowed Jmol to
support 2D to 3D conversion of structures (through energy
minimisation). Similarly, efficient Jmol code for atom-atom rebonding
has been ported to the CDK.

Another collaborative initiative between Blue Obelisk projects was the establishment in May 2008 of
the ChemiSQL project. This brought together the developers of several
open source chemistry database cartridges (PgChem, MyChem, OrChem and
more recently Bingo) with a view to making their database APIs more
similar and collaborating on benchmark datasets for assessing
performance. For two of these projects, PgChem and MyChem, which are both based on
Open Babel, there is the additional possibility of working together on a shared
codebase.

In the area of cheminformatics toolkits, two of the existing toolkits
Open Babel and RDKit are planning to work together on a common
underlying framework called MolCore.\cite{WebMolCore} This project is still in the
planning stage, but if it is a success it will mean that the the two
libraries will be interoperable (while retaining their existing focus)
but also that the cost of maintaining the code will be shared among
more developers, freeing time for the development of new features.

One of the goals of the Blue Obelisk is to promote interoperability in chemical
informatics. When barriers exist to moving chemical data between
different software, the community becomes fragmented and there is
the danger of vendor lock-in (where users are constrained to using
a particular software, a situation which puts them at a
disadvantage). This applies as much to Open Source software as to
proprietary software. Cinfony is a project (first release in May 2008)
whose goal is to tackle this problem in the area of cheminformatics
toolkits \cite{OBoyleCinfony2008}.
It is a Python library that enables Open Babel, the CDK, and RDKit to
be used using the same API; this makes it easy, for example, to read a
molecule using Open Babel, calculate descriptors using the CDK and
create a depiction using RDKit.

  \subsection*{Remaining challenges}

Accuracy. Very often software work at the 95\% to 98\% level. The
variety of chemical structures is such that with a large enough
dataset, 'unusual' structures are always found which may be mishandled
by software. Given that much of the development of open source
software is unfunded, and relies heavily on developer motivation, it
is understandable that working on the final N\% of problem structures
(which may require substantial work) is not the most exciting of
tasks. Putting a bounty system in place may be useful for these
situations where a researcher needs a problem to be fixed because it
affects a particular dataset of interest.

Performance. There is a famous quote by Knuth that "Premature
optimisation is root of all evil (or something)". Given that compute
time is cheap, ....

\section*{Chemical Structure Registration Systems, Databases}
Databases? (XXX CS: A couple of database-related paragraphs could go here XXX)

XXX the paragraph below on OrChem needs some work
Registration, indexing and searching of chemical structures in relational databases is one of the core areas of cheminformatics. 
A number of structure registration systems have been published in the last five years, exploiting the fact that 
free cheminformatics toolkits such as OpenBabel and the CDK were available. 
OrChem, for example, is an extension for the Oracle 11G database that adds registration and indexing of chemical structures to support fast substructure and similarity searching. The cheminformatics functionality is provided by the Chemistry Development Kit. OrChem provides similarity searching with response times in the order of seconds for databases with millions of compounds, depending on a given similarity cut-off. For substructure searching, it can make use of multiple processor cores on today's powerful database servers to provide fast response times in equally large data sets.
OrChem is free software and can be redistributed and/or modified under the terms of the GNU Lesser General Public License as published by the Free Software Foundation. All software is available via http://orchem.sourceforge.net.

\section*{Open Standards}
  \subsection*{Progress}

The IUPAC InChI identifier is a non-proprietary and unique identifier
for chemical substances designed to enable linking of diverse data
compilations. Although its development predates the Blue Obelisk,
software such as Open Babel has included InChI support since 2005.
Since the official InChI implementation is in C, it is difficult to
access from the other widely used language for cheminformatics
toolkits, Java. The Blue Obelisk project JNI-InChI has been set up to
solve this problem by using the Java Native Interface to link the
InChI binary to Java. In this way, it promotes the wider adoption of
this standard identifier by the chemistry community.

Also, Indigo is going to provide its own plugin with an interface to
the InChI implementation in May 2011. This will enable developers to
calculate InChI and InChIKey-s within Python and .NET projects (not to
mention Java), eliminating the trouble of compiling and using binary
modules directly.

    \subsubsection*{OpenSMILES}

One of the most widely used ways to store chemical structures is the
SMILES format (or SMILES string). This is a linear notation depicted
by Daylight Information Systems that describes the connection table
of a molecule and may optionally encode chirality. Its popularity
stems from the fact that it is a compact representation of the
chemical structure that is human readable and writable, and is
convenient to manipulate (e.g. to include in spreadsheets, or copy
from a Wikipedia article).

Despite its widespread use, a formal
definition of the language did not exist beyond Daylight's SMILES
Theory Manual and tutorials. This caused some confusion in the
implementation and interpretation of corner cases, for example the
handling of cis/trans bond symbols at ring closures. In 2007, Craig
James (eMolecules) initiated work on the OpenSMILES specification, a
complete specification of the SMILES language as an Open Standard
developed through a community process. The specification is largely
complete and contains guidelines on reading SMILES, a formal
grammar, recommendations on standard forms when writing SMILES, as
well as proposed extensions.

Recently proposed CurlySMILES\cite{CurlySMILES} is an extension of the
SMILES notation, which allows to define crystals structures, polymers,
electron delocalisation charges, molecule interactions, and many other
features absent in the initial SMILES specification. There has been
discussions about including parts of the CurlySMILES notation into
OpenSMILES, especially polymers.

\subsubsection*{CML}
?


\subsubsection*{QSAR-ML}
The field of QSAR has long been hampered by the lack of open standards, which makes it difficult to share and reproduce descriptor calculations and analyses. QSAR-ML was recently proposed as an open standard for exchanging QSAR datasets~\cite{Spjuth:2010uq}. A dataset in QSAR-ML includes the chemical structures (preferable described in CML) with InChI to protect integrity, chemical descriptors by linking to the Blue Obelisk Descriptor Ontology~\cite{bodo}, response values, units, and versioned descriptor implementations to allow for integrating several descriptor software in the same calculation. Hence, a dataset described in QSAR-ML is completely reproducible. To allow for easy setup of QSAR-ML compliant datasets, a plugin for Bioclipse was created with graphical interfaces that can be used to set up QSAR datasets and perform calculations. Descriptor implementations were initially available from CDK and JOELib, as well as via remote web services such as XMPP~\cite{Wagener:2009uq}.


\subsubsection*{Others...?}

Pistoia Alliance (and others?) should be mentioned at this point

  \subsection*{Remaining challenges}

A core requirement for chemical structure databases and chemical
registration systems in general is the notion of structure
standardisation.  That is,  for a given input structure, multiple
representations should be converted to one canonical form. 
Structure canonicalisation routines partially address this aspect,
converting multiple alternative topologies to a single canonical
form. However, the problem of standardisation is broader than just
topological canonicalisation. Features that must be considered include
\begin{itemize}
\item topological canonicalisation
\item handling of charges
\item tautomer enumeration and canonicalisation
\item normalisation of functional groups
\end{itemize}
Currently, most of the individual components of a `standardisation
pipeline' can be implemented using BO tools. The larger problem is
that there is no agreed upon list of steps for a standardisation
process. While some specifications have been published (e.g., PubChem)
and some standardisation services and tools are available (PubChem
provides an online service to standardise molecules and the NCGC
provides a stand alone tool) each group has their own set of rules. A
common reference specification for standardisation would be of immense
value in interoperability between structure repositories as well as
between toolkits (though the latter is still confounded by differences
in lower level cheminformatic features such as aromaticity models).

We have already discussed the development of an Open SMILES standard.
While much progress has been made towards a complete specification,
more remains to be done before this can be considered finished. After
that point, the next logical step would be to start work on a standard
for the SMARTS language, the extension to SMILES that specifies
patterns that match chemical substructures.

\section*{Open Data}
  \subsection*{Progress}

A considerable stumbling block in advocating the release of scientific
data as Open Data has been how exactly to define Open. A major step
forward was the launch in 2010 of the Panton Principles for Open Data
in Science \cite{WebPanton}. This formalises the idea that Open Data maximises the
possibility of reuse and repurposing, the fundamental basis
of how science works. These principles recommend that published data
be licensed explicitly, and preferably under CC0 (Creative Commons `No
Rights Reserved', also known as CCZero) \cite{WebCC0}. This license allows others to use the 
data for any purpose whatsoever without any barriers. Other licenses
compatible with the Panton Principles include the
Open Data Commons Public Domain Dedication and Licence (PDDL), the
Open Data Commons Attribution License, and the
Open Data Commons Open Database License (ODbL) (see
http://www.opendefinition.org/licenses/\#Data).

Despite this positive news, little chemical data has become
available from the traditional chemical fields of organic,
inorganic, solid state chemistry. Table~2 lists a few notable
exceptions: ChemPedia (a now discontinued crowd-sourcing project),
CrystalEye (http://wwmm.ch.cam.ac.uk/crystaleye/),
and the Open Notebook Science Solubility
data~\cite{ONS2010}. There is also data available using licenses
not compatible with the Panton Principles, but where the user
is allows to modify and redistribute the data. A new data
set in this category is the data from the ChEMBL database, 
which is available under the Creative Commons Share-Alike
Attribution license~\cite{Overington2009}.

Importantly, publishing data as CC0 is become easier now that
website are becoming available to simplify publishing data. Two
projects that can be mentioned in this context are FigShare
(http://figshare.com/), where the data behind unpublished figures
can be hosted, and Dryad (http://datadryad.org/) where data
behind publications can be hosted. Initiatives like this make
it possible to host small amounts of data, and those combined
are expected to become soon a substantial knowledge base.

  \subsection*{Remaining challenges}

\section*{Other areas of activity}
Web services? XMPP services? (NIH, CDK and ChemSpider)


\subsection*{Getting the message out}

One recent development
that attempts to address this issue was the establishment in Apr 2010 of a
question-and-answer (Q\&A)
website related to Blue Obelisk projects and themes at
http://blueobelisk.shapado.com (see Figure~3). This is a website in the
style of Stack Overflow that encourages high quality answers (and
questions) through the use of a voting system. In the year since it
was established, over
200 users have registered, many of whom had no previous involvement
with the Blue Obelisk, showing that the Q\&A website is a nice complement
to earlier existing channels of communication.


Books (Angel, Egon, OB)

%%%%%%%%%%%%%%%%%%%%%%
\section*{Conclusions}

We have shown that the Blue Obelisk has been very successful
in bringing together researchers and developers with common interests
in ODOSOS, leading to development of many useful resources freely
available to the chemistry community. Figure 2 shows how the various
Blue Obelisk projects collaborate. But how best to engage with the
wider chemistry community outside of the Blue Obelisk remains an open
question. If the Blue Obelisk is truly to make an impact,
then an attempt must be made to reach beyond the subscribers to the
BO mailing list and blogs of members.

We hope to see this involvement between the Blue Obelisk and the wider
community grow in the future. To this end, we encourage the reader to
visit http://blueobelisk.org, send a message to our mailing list,
investigate related projects or read our blogs.


%%%%%%%%%%%%%%%%%%%%%%%%%%%%%%%%
\section*{Authors contributions}
   Charles Darwin did all the work. The others stole the glory. 
    

%%%%%%%%%%%%%%%%%%%%%%%%%%%
\section*{Acknowledgements}
  \ifthenelse{\boolean{publ}}{\small}{}
  Thanks to everyone.


 
%%%%%%%%%%%%%%%%%%%%%%%%%%%%%%%%%%%%%%%%%%%%%%%%%%%%%%%%%%%%%
%%                  The Bibliography                       %%
%%                                                         %%              
%%  Bmc_article.bst  will be used to                       %%
%%  create a .BBL file for submission, which includes      %%
%%  XML structured for BMC.                                %%
%%                                                         %%
%%                                                         %%
%%  Note that the displayed Bibliography will not          %% 
%%  necessarily be rendered by Latex exactly as specified  %%
%%  in the online Instructions for Authors.                %% 
%%                                                         %%
%%%%%%%%%%%%%%%%%%%%%%%%%%%%%%%%%%%%%%%%%%%%%%%%%%%%%%%%%%%%%


{\ifthenelse{\boolean{publ}}{\footnotesize}{\small}
 \bibliographystyle{bmc_article}  % Style BST file
  \bibliography{websites,paper} }     % Bibliography file (usually '*.bib' ) 

%%%%%%%%%%%

\ifthenelse{\boolean{publ}}{\end{multicols}}{}

%%%%%%%%%%%%%%%%%%%%%%%%%%%%%%%%%%%
%%                               %%
%% Figures                       %%
%%                               %%
%% NB: this is for captions and  %%
%% Titles. All graphics must be  %%
%% submitted separately and NOT  %%
%% included in the Tex document  %%
%%                               %%
%%%%%%%%%%%%%%%%%%%%%%%%%%%%%%%%%%%

%%
%% Do not use \listoffigures as most will included as separate files

\section*{Figures}
  \subsection*{Figure 1 - Blue Obelisk logo}

  \subsection*{Figure 2 - Dependency diagram of Blue Obelisk projects.}
      Each block represents a project. Square blocks show Open Data, ovals are Open Source,
      and diamonds are Open Standards. Colors represent license: LGPL is green, GPL is orange,
      and BSD is blue.

  \subsection*{Figure 3 - Screenshot of the Blue Obelisk eXchange Question
    and Answer website.}


%%%%%%%%%%%%%%%%%%%%%%%%%%%%%%%%%%%
%%                               %%
%% Tables                        %%
%%                               %%
%%%%%%%%%%%%%%%%%%%%%%%%%%%%%%%%%%%

%% Use of \listoftables is discouraged.
%%
\section*{Tables}
  \subsection*{Table 1 - Blue Obelisk Open Source software projects}
    (Description if necessary XXXXXXXXXXXXXXX. Add citations to project names.)
    \par \mbox{}
    \par
    \mbox{
      \begin{tabular}{|c|c|c|}
        %% \hline \multicolumn{3}{|c|}{My Table}\\ \hline
        \hline Name & Website & Description? or Lead Developer? \\ \hline
        \multicolumn{3}{|c|}{Cheminformatics toolkits} \\ \hline
        Chemistry Development Kit (CDK) \cite{Steinbeck2003, Steinbeck2006} & http://cdk.sf.net & XXXX  \\ \hline
        Cinfony & http://cinfony.googlecode.com & Noel O'Boyle, Python interface to toolkits \\ \hline
        Indigo & http://ggasoftware.com/opensource/indigo & GGA Software \\ \hline
        Open Babel & http://openbabel.org & Geoffrey Hutchison et al \\ \hline
        RDKit & http://rdkit.org & Greg Landrum \\ \hline
        ChemDoodle Web Components & http://web.chemdoodle.com & iChemLabs \\ \hline
        \multicolumn{3}{|c|}{Integration} \\ \hline
        CDK-Taverna \cite{Kuhn:2010p4001} & http://cdk-taverna.blah.XXX & Christoph Steinbeck, Workflow \\ \hline
        A3 & ..  & .  \\ \hline
        \multicolumn{3}{|c|}{Interconversion} \\ \hline
        OSRA & http://osra.sf.net & Igor Filippov, Image to structure \\ \hline
        OPSIN & http://opsin.ch.cam.ac.uk & Daniel Lowe, Name to structure \\ \hline
        A3 & ..  & .  \\ \hline
        \multicolumn{3}{|c|}{Structure Databases} \\ \hline
        OrChem \cite{RijnbeekS10} & http://orchem.sourceforge.net & Christoph Steinbeck, Oracle-based Chemical Database Engine \\ \hline
        MyChem  & http://XXX.sourceforge.net & Jerome Pansanel, MySQL-based Chemical Database Engine \\ \hline
        PGChem  & http://XXX.sourceforge.net & Ernst-Georg Schmid, PostgreSQL-based Chemical Database Engine \\ \hline
        Bingo  & http://XXX.XXX.XXX & Dmitry Pavlov, Oracle-based Chemical Database Engine \\ \hline

      \end{tabular}
      }
  \subsection*{Table 2 - Open Data in chemistry.}
    Overview of major open chemical data available under a license or waiver
    compatible with the Panton Principles.
    \par \mbox{}
    \par
    \mbox{
\begin{tabular}{|l|l|l|}
  %% \hline \multicolumn{3}{|c|}{My Table}\\ \hline
  \hline Name & License/Waiver & Description \\ \hline
  ChemPedia & CC0 & Crowd-sourced chemical names. Project discontinued. \\ \hline
  CrystalEye     & PPDL & Crystal structures from primary literature. \\ \hline
  ONS Solubility & CC0 & Solubility data for various solvents. \\ \hline
\end{tabular}
      }



%%%%%%%%%%%%%%%%%%%%%%%%%%%%%%%%%%%
%%                               %%
%% Additional Files              %%
%%                               %%
%%%%%%%%%%%%%%%%%%%%%%%%%%%%%%%%%%%

\section*{Additional Files}
  \subsection*{Additional file 1 --- Sample additional file title}
    Additional file descriptions text (including details of how to
    view the file, if it is in a non-standard format or the file extension).  This might
    refer to a multi-page table or a figure.

  \subsection*{Additional file 2 --- Sample additional file title}
    Additional file descriptions text.


\end{bmcformat}
\end{document}







