%% BioMed_Central_Tex_Template_v1.05
%%                                      %
%  bmc_article.tex            ver: 1.05 %
%                                       %


%%%%%%%%%%%%%%%%%%%%%%%%%%%%%%%%%%%%%%%%%
%%                                     %%
%%  LaTeX template for BioMed Central  %%
%%     journal article submissions     %%
%%                                     %%
%%         <27 January 2006>           %%
%%                                     %%
%%                                     %%
%% Uses:                               %%
%% cite.sty, url.sty, bmc_article.cls  %%
%% ifthen.sty. multicol.sty		      %%
%%									   %%
%%                                     %%
%%%%%%%%%%%%%%%%%%%%%%%%%%%%%%%%%%%%%%%%%


%%%%%%%%%%%%%%%%%%%%%%%%%%%%%%%%%%%%%%%%%%%%%%%%%%%%%%%%%%%%%%%%%%%%%
%%                                                                 %%	
%% For instructions on how to fill out this Tex template           %%
%% document please refer to Readme.pdf and the instructions for    %%
%% authors page on the biomed central website                      %%
%% http://www.biomedcentral.com/info/authors/                      %%
%%                                                                 %%
%% Please do not use \input{...} to include other tex files.       %%
%% Submit your LaTeX manuscript as one .tex document.              %%
%%                                                                 %%
%% All additional figures and files should be attached             %%
%% separately and not embedded in the \TeX\ document itself.       %%
%%                                                                 %%
%% BioMed Central currently use the MikTex distribution of         %%
%% TeX for Windows) of TeX and LaTeX.  This is available from      %%
%% http://www.miktex.org                                           %%
%%                                                                 %%
%%%%%%%%%%%%%%%%%%%%%%%%%%%%%%%%%%%%%%%%%%%%%%%%%%%%%%%%%%%%%%%%%%%%%


\NeedsTeXFormat{LaTeX2e}[1995/12/01]
\documentclass[10pt]{bmc_article}    



% Load packages
\usepackage{cite} % Make references as [1-4], not [1,2,3,4]
\usepackage{url}  % Formatting web addresses  
\usepackage{ifthen}  % Conditional 
\usepackage{multicol}   %Columns
\usepackage[utf8]{inputenc} %unicode support
%\usepackage[applemac]{inputenc} %applemac support if unicode package fails
%\usepackage[latin1]{inputenc} %UNIX support if unicode package fails
\urlstyle{rm}
 
 
%%%%%%%%%%%%%%%%%%%%%%%%%%%%%%%%%%%%%%%%%%%%%%%%%	
%%                                             %%
%%  If you wish to display your graphics for   %%
%%  your own use using includegraphic or       %%
%%  includegraphics, then comment out the      %%
%%  following two lines of code.               %%   
%%  NB: These line *must* be included when     %%
%%  submitting to BMC.                         %% 
%%  All figure files must be submitted as      %%
%%  separate graphics through the BMC          %%
%%  submission process, not included in the    %% 
%%  submitted article.                         %% 
%%                                             %%
%%%%%%%%%%%%%%%%%%%%%%%%%%%%%%%%%%%%%%%%%%%%%%%%%                     


\def\includegraphic{}
\def\includegraphics{}



\setlength{\topmargin}{0.0cm}
\setlength{\textheight}{21.5cm}
\setlength{\oddsidemargin}{0cm} 
\setlength{\textwidth}{16.5cm}
\setlength{\columnsep}{0.6cm}

\newboolean{publ}

%%%%%%%%%%%%%%%%%%%%%%%%%%%%%%%%%%%%%%%%%%%%%%%%%%
%%                                              %%
%% You may change the following style settings  %%
%% Should you wish to format your article       %%
%% in a publication style for printing out and  %%
%% sharing with colleagues, but ensure that     %%
%% before submitting to BMC that the style is   %%
%% returned to the Review style setting.        %%
%%                                              %%
%%%%%%%%%%%%%%%%%%%%%%%%%%%%%%%%%%%%%%%%%%%%%%%%%%
 

%Review style settings
\newenvironment{bmcformat}{\begin{raggedright}\baselineskip20pt\sloppy\setboolean{publ}{false}}{\end{raggedright}\baselineskip20pt\sloppy}

%Publication style settings
%\newenvironment{bmcformat}{\fussy\setboolean{publ}{true}}{\fussy}



% Begin ...
\begin{document}
\begin{bmcformat}


%%%%%%%%%%%%%%%%%%%%%%%%%%%%%%%%%%%%%%%%%%%%%%
%%                                          %%
%% Enter the title of your article here     %%
%%                                          %%
%%%%%%%%%%%%%%%%%%%%%%%%%%%%%%%%%%%%%%%%%%%%%%

\title{The Blue Obelisk five years on}
 
%%%%%%%%%%%%%%%%%%%%%%%%%%%%%%%%%%%%%%%%%%%%%%
%%                                          %%
%% Enter the authors here                   %%
%%                                          %%
%% Ensure \and is entered between all but   %%
%% the last two authors. This will be       %%
%% replaced by a comma in the final article %%
%%                                          %%
%% Ensure there are no trailing spaces at   %% 
%% the ends of the lines                    %%     	
%%                                          %%
%%%%%%%%%%%%%%%%%%%%%%%%%%%%%%%%%%%%%%%%%%%%%%


\author{Charles A Darwin\correspondingauthor$^{1,2}$%
       \email{Charles A Darwin\correspondingauthor - charles@londonzoo.co.uk}%
      \and 
         Egon L Willighagen$^3$%
         \email{Egon L Willighagen - egon.willighagen@ki.se}
     \and 
         Rajarshi Guha$^4$%
         \email{Rajarshi Guha - guhar@mail.nih.gov}
     \and 
         Noel M O'Boyle$^5$%
         \email{Noel M O'Boyle - n.oboyle@ucc.ie}
      \and
         Jane E Doe\correspondingauthor$^2$%
         \email{Jane E Doe\correspondingauthor - jane.e.doe@cambridge.co.uk}
      }
      

%%%%%%%%%%%%%%%%%%%%%%%%%%%%%%%%%%%%%%%%%%%%%%
%%                                          %%
%% Enter the authors' addresses here        %%
%%                                          %%
%%%%%%%%%%%%%%%%%%%%%%%%%%%%%%%%%%%%%%%%%%%%%%

\address{%
    \iid(1)Life Sciences Department, Kings College London, Cornwall House,%
        Waterloo Road, London, UK\\
    \iid(2)Department of Zoology, Cambridge, Waterloo Road, London, UK\\
    \iid(3)Division of Molecular Toxicology, Institute of Environmental Medicine, %
        Nobels vaeg 13, Karolinska Institutet, 171 77 Stockholm, Sweden
    \iid(5)Analytical and Biological Chemistry Research Facility, Cavanagh Pharmacy Building, University College Cork, College Road, Cork, Co. Cork, Ireland
}%

\maketitle

%%%%%%%%%%%%%%%%%%%%%%%%%%%%%%%%%%%%%%%%%%%%%%
%%                                          %%
%% The Abstract begins here                 %%
%%                                          %%
%% The Section headings here are those for  %%
%% a Research article submitted to a        %%
%% BMC-Series journal.                      %%  
%%                                          %%
%% If your article is not of this type,     %%
%% then refer to the Instructions for       %%
%% authors on http://www.biomedcentral.com  %%
%% and change the section headings          %%
%% accordingly.                             %%   
%%                                          %%
%%%%%%%%%%%%%%%%%%%%%%%%%%%%%%%%%%%%%%%%%%%%%%


\begin{abstract}
        % Do not use inserted blank lines (ie \\) until main body of text.
        \paragraph*{Background:} Text for this section of the abstract. 
      
        \paragraph*{Results:} Text for this section of the abstract \ldots

        \paragraph*{Conclusions:} Text for this section of the abstract \ldots
\end{abstract}



\ifthenelse{\boolean{publ}}{\begin{multicols}{2}}{}




%%%%%%%%%%%%%%%%%%%%%%%%%%%%%%%%%%%%%%%%%%%%%%
%%                                          %%
%% The Main Body begins here                %%
%%                                          %%
%% The Section headings here are those for  %%
%% a Research article submitted to a        %%
%% BMC-Series journal.                      %%  
%%                                          %%
%% If your article is not of this type,     %%
%% then refer to the instructions for       %%
%% authors on:                              %%
%% http://www.biomedcentral.com/info/authors%%
%% and change the section headings          %%
%% accordingly.                             %% 
%%                                          %%
%% See the Results and Discussion section   %%
%% for details on how to create sub-sections%%
%%                                          %%
%% use \cite{...} to cite references        %%
%%  \cite{koon} and                         %%
%%  \cite{oreg,khar,zvai,xjon,schn,pond}    %%
%%  \nocite{smith,marg,hunn,advi,koha,mouse}%%
%%                                          %%
%%%%%%%%%%%%%%%%%%%%%%%%%%%%%%%%%%%%%%%%%%%%%%




%%%%%%%%%%%%%%%%
%% Background %%
%%
\section*{Background}
The Blue Obelisk group was established in 2005 at the
229\textsuperscript{th} National Meeting of the American Chemistry
Society as a response to the lack of open data, open standards and
open source (ODOSOS) in chemistry. While other scientific disciplines
such as physics, biology and astronomy (to name a few) were embracing
new ways of doing science and reaping the benefits of community
efforts, there was little if any innovation in the field of chemistry
and scientific progress was actively hampered by the lack of access to
data and tools.

The goals of the Blue Obelisk group were described in Guha et al.\cite{guha2006}. but are briefly summarised here...


%%%%%%%%%%%%%%%%%%%%%%%%%%%%
%% Main section %%
%%
\section*{Open Source}
  \subsection*{Progress}

...something about toolkits...

Although feature-rich and robust cheminformatics toolkits are useful in and of themselves, they can also be seen as providing a base layer on which tools and applications can be built. Now that these toolkits have reached an acceptable level of stability and functionality, an increasing amount of software is being developed using these toolkits. As an example of the interaction between the commercial and open source worlds, Silicos (a Belgian company that provides services in the area of cheminformatics XXXX rephrase), has released several command line applications based on Open Babel; Pharao for pharmacophore generation (Sept 2010?), Sieve for filtering by molecular property, ..... 

It is worth noting that Pharao is the first Open Source pharmacophore program (\emph{Is this a reasonable statement?}) (although pharmacophore-related functionality has been available in the CDK and RDKit for some time).

  \subsubsection*{Collaboration and interoperability}

One of the effects of the Blue Obelisk has been to bring developers together from different Open Source chemistry projects so that they look for opportunities to collaborate rather than compete, and to leverage work done by other projects to avoid duplication of effort.

As an example of this, when in March 2008 the Jmol development team were looking to add support for energy minimisation, rather than implement a forcefield from scratch they ported the UFF forcefield implementation from Open Babel to Jmol. This code has allowed Jmol to support 2D to 3D conversion of structures (through energy minimisation).

In the area of cheminformatics tookits, two of the existing toolkits Open Babel and RDKit are planning to work together on a common underlying framework called MolCore. This project is still in the planning stage, but if it is a success it will mean that the the two libraries will be interoperable (while retaining their existing focus) but also that the cost of maintaining the code will be shared among more developers, freeing time for the development of new features.

One of the goals of the Blue Obelisk is to promote interoperability in chemical informatics. When barriers exist to moving chemical data between different software, the community becomes fragmented and there is the danger of vendor lock-in (where users are constrained to using a particular software, a situation which puts them at a disadvantage). 

...cinfony...

  \subsection*{Remaining challenges}

Accuracy. Very often software work at the 95\% to 98\% level. The variety of chemical structures is such that with a large enough dataset, 'unusual' structures are always found which may be mishandled by software. Given that much of the development of open source software is unfunded, and relies heavily on developer motivation, it is understandable that working on the final N\% of problem structures (which may require substantial work) is not the most exciting of tasks. Putting a bounty system in place may be useful for these situations where a researcher needs a problem to be fixed because it affects a particular dataset of interest.

Performance. There is a famous quote by Knuth that "Premature optimization is root of all evil (or something)". Given that compute time is cheap, ....

Databases?

\section*{Open Standards}
  \subsection*{Progress}

    \subsubsection*{OpenSMILES}

One of the most widely used ways to store chemical structures is the SMILES format (or SMILES string). This is a linear notation depicted by Daylight Information Systems that describes the connection table of a molecule and may optionally encode chirality. Its popularity stems from the fact that it is a compact representation of the chemical structure that is human readable and writable, and is convenient to manipulate (e.g. to include in spreadsheets, or copy from a Wikipedia article).



InChI

CML?

Pistoia Alliance (and others?) should be mentioned at this point

  \subsection*{Remaining challenges}

OpenSMARTS.

A core requirement for chemical structure databases and chemical
registration systems in general is the notion of structure
standardization.  That is,  for a given input structure, multiple
representations should be converted to one canonical form. 
Structure canonicalization routines partially address this aspect,
converting multiple alternative topologies to a single canonical
form. However, the problem of standardization is broader than just
topological canonicalization. Features that must be considered include
\begin{itemize}
\item topological canonicalization
\item handling of charges
\item tautomer enumeration and canonicalization
\item normalization of functional groups
\end{itemize}
Currently, most of the individual components of a ``standardization
pipeline'' can be implemented using BO tools. The larger problem is
that there is no agreed upon list of steps for a standardization
process. While some specifications have been published (e.g., Pubchem)
and some standardization services and tools are available (Pubchem
provides an online service to standaridize molecules and the NCGC
provides a stand alone tool) each group has their own set of rules. A
common reference specification for standardization would be of immense
value in interoperability between structure repositories as well as
between toolkits (though the latter is still confounded by differences
in lower level cheminformatic features such as aromaticity models).

\section*{Open Data}
  \subsection*{Progress}
  \subsection*{Remaining challenges}

%%%%%%%%%%%%%%%%%%%%%%%%%%%%
%% Results and Discussion %%
%%
\section*{Results and Discussion}
  \subsection*{Results sub-heading}
    \subsubsection*{This is a sub-sub-heading}
      Sub-sub-sub-headings are made with the \textsl{\\subsubsection} command. \pb
      pb at end of lines ensures correct paragraph spacing.\pb
	  Text for this sub-sub-section \ldots
    \subsubsection*{Another sub-sub-sub-heading}
      Text for this sub-sub-section \ldots

  \subsection*{Another results sub-heading}
    Text for this sub-section \ldots

  \subsection*{Yet another results sub-heading}
    Text for this sub-section.  More results \ldots


    

%%%%%%%%%%%%%%%%%%%%%%
\section*{Conclusions}
  Text for this section \ldots


%%%%%%%%%%%%%%%%%%%%%%%%%%%%%%%%
\section*{Authors contributions}
   Charles Darwin did all the work. The others stole the glory. 

    

%%%%%%%%%%%%%%%%%%%%%%%%%%%
\section*{Acknowledgements}
  \ifthenelse{\boolean{publ}}{\small}{}
  Thanks to everyone.


 
%%%%%%%%%%%%%%%%%%%%%%%%%%%%%%%%%%%%%%%%%%%%%%%%%%%%%%%%%%%%%
%%                  The Bibliography                       %%
%%                                                         %%              
%%  Bmc_article.bst  will be used to                       %%
%%  create a .BBL file for submission, which includes      %%
%%  XML structured for BMC.                                %%
%%                                                         %%
%%                                                         %%
%%  Note that the displayed Bibliography will not          %% 
%%  necessarily be rendered by Latex exactly as specified  %%
%%  in the online Instructions for Authors.                %% 
%%                                                         %%
%%%%%%%%%%%%%%%%%%%%%%%%%%%%%%%%%%%%%%%%%%%%%%%%%%%%%%%%%%%%%


{\ifthenelse{\boolean{publ}}{\footnotesize}{\small}
 \bibliographystyle{bmc_article}  % Style BST file
  \bibliography{paper} }     % Bibliography file (usually '*.bib' ) 

%%%%%%%%%%%

\ifthenelse{\boolean{publ}}{\end{multicols}}{}

%%%%%%%%%%%%%%%%%%%%%%%%%%%%%%%%%%%
%%                               %%
%% Figures                       %%
%%                               %%
%% NB: this is for captions and  %%
%% Titles. All graphics must be  %%
%% submitted separately and NOT  %%
%% included in the Tex document  %%
%%                               %%
%%%%%%%%%%%%%%%%%%%%%%%%%%%%%%%%%%%

%%
%% Do not use \listoffigures as most will included as separate files

\section*{Figures}
  \subsection*{Figure 1 - Sample figure title}
      A short description of the figure content
      should go here.

  \subsection*{Figure 2 - Sample figure title}
      Figure legend text.



%%%%%%%%%%%%%%%%%%%%%%%%%%%%%%%%%%%
%%                               %%
%% Tables                        %%
%%                               %%
%%%%%%%%%%%%%%%%%%%%%%%%%%%%%%%%%%%

%% Use of \listoftables is discouraged.
%%
\section*{Tables}
  \subsection*{Table 1 - Blue Obelisk Open Source software projects}
    (Description if necessary XXXXXXXXXXXXXXX. Add citations to project names.)
    \par \mbox{}
    \par
    \mbox{
      \begin{tabular}{|c|c|c|}
        %% \hline \multicolumn{3}{|c|}{My Table}\\ \hline
        \hline Name & Website & Description? or Lead Developer? \\ \hline
        \multicolumn{3}{|c|}{Cheminformatics toolkits} \\ \hline
        Chemistry Development Kit & http://cdk.sf.net & XXXX  \\ \hline
        Cinfony & http://cinfony.googlecode.com & Noel O'Boyle, Python interface to toolkits \\ \hline
        Indigo & http://ggasoftware.com/opensource/indigo & GGA Software \\ \hline
        Open Babel & http://openbabel.org & Geoffrey Hutchison et al \\ \hline
        RDKit & http://rdkit.org & Greg Landrum \\ \hline
        \multicolumn{3}{|c|}{Interconversion} \\ \hline
        OSRA & http://osra.sf.net & Igor Filippov, Image to structure \\ \hline
        A3 & ..  & .  \\ \hline
      \end{tabular}
      }
  \subsection*{Table 2 - Sample table title}
    Large tables are attached as separate files but should
    still be described here.



%%%%%%%%%%%%%%%%%%%%%%%%%%%%%%%%%%%
%%                               %%
%% Additional Files              %%
%%                               %%
%%%%%%%%%%%%%%%%%%%%%%%%%%%%%%%%%%%

\section*{Additional Files}
  \subsection*{Additional file 1 --- Sample additional file title}
    Additional file descriptions text (including details of how to
    view the file, if it is in a non-standard format or the file extension).  This might
    refer to a multi-page table or a figure.

  \subsection*{Additional file 2 --- Sample additional file title}
    Additional file descriptions text.


\end{bmcformat}
\end{document}







