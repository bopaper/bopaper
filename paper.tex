%% BioMed_Central_Tex_Template_v1.05
%%                                      %
%  bmc_article.tex            ver: 1.05 %
%                                       %


%%%%%%%%%%%%%%%%%%%%%%%%%%%%%%%%%%%%%%%%%
%%                                     %%
%%  LaTeX template for BioMed Central  %%
%%     journal article submissions     %%
%%                                     %%
%%         <27 January 2006>           %%
%%                                     %%
%%                                     %%
%% Uses:                               %%
%% cite.sty, url.sty, bmc_article.cls  %%
%% ifthen.sty. multicol.sty		      %%
%%									   %%
%%                                     %%
%%%%%%%%%%%%%%%%%%%%%%%%%%%%%%%%%%%%%%%%%


%%%%%%%%%%%%%%%%%%%%%%%%%%%%%%%%%%%%%%%%%%%%%%%%%%%%%%%%%%%%%%%%%%%%%
%%                                                                 %%	
%% For instructions on how to fill out this Tex template           %%
%% document please refer to Readme.pdf and the instructions for    %%
%% authors page on the biomed central website                      %%
%% http://www.biomedcentral.com/info/authors/                      %%
%%                                                                 %%
%% Please do not use \input{...} to include other tex files.       %%
%% Submit your LaTeX manuscript as one .tex document.              %%
%%                                                                 %%
%% All additional figures and files should be attached             %%
%% separately and not embedded in the \TeX\ document itself.       %%
%%                                                                 %%
%% BioMed Central currently use the MikTex distribution of         %%
%% TeX for Windows) of TeX and LaTeX.  This is available from      %%
%% http://www.miktex.org                                           %%
%%                                                                 %%
%%%%%%%%%%%%%%%%%%%%%%%%%%%%%%%%%%%%%%%%%%%%%%%%%%%%%%%%%%%%%%%%%%%%%


\NeedsTeXFormat{LaTeX2e}[1995/12/01]
\documentclass[10pt]{bmc_article}    



% Load packages
\usepackage{cite} % Make references as [1-4], not [1,2,3,4]
\usepackage{url}  % Formatting web addresses  
\usepackage{ifthen}  % Conditional 
\usepackage{multicol}   %Columns
\usepackage[utf8]{inputenc} %unicode support
%\usepackage[applemac]{inputenc} %applemac support if unicode package fails
%\usepackage[latin1]{inputenc} %UNIX support if unicode package fails
\urlstyle{rm}
 
 
%%%%%%%%%%%%%%%%%%%%%%%%%%%%%%%%%%%%%%%%%%%%%%%%%	
%%                                             %%
%%  If you wish to display your graphics for   %%
%%  your own use using includegraphic or       %%
%%  includegraphics, then comment out the      %%
%%  following two lines of code.               %%   
%%  NB: These line *must* be included when     %%
%%  submitting to BMC.                         %% 
%%  All figure files must be submitted as      %%
%%  separate graphics through the BMC          %%
%%  submission process, not included in the    %% 
%%  submitted article.                         %% 
%%                                             %%
%%%%%%%%%%%%%%%%%%%%%%%%%%%%%%%%%%%%%%%%%%%%%%%%%                     


\def\includegraphic{}
\def\includegraphics{}



\setlength{\topmargin}{0.0cm}
\setlength{\textheight}{21.5cm}
\setlength{\oddsidemargin}{0cm} 
\setlength{\textwidth}{16.5cm}
\setlength{\columnsep}{0.6cm}

\newboolean{publ}

%%%%%%%%%%%%%%%%%%%%%%%%%%%%%%%%%%%%%%%%%%%%%%%%%%
%%                                              %%
%% You may change the following style settings  %%
%% Should you wish to format your article       %%
%% in a publication style for printing out and  %%
%% sharing with colleagues, but ensure that     %%
%% before submitting to BMC that the style is   %%
%% returned to the Review style setting.        %%
%%                                              %%
%%%%%%%%%%%%%%%%%%%%%%%%%%%%%%%%%%%%%%%%%%%%%%%%%%
 

%Review style settings
\newenvironment{bmcformat}{\begin{raggedright}\baselineskip20pt\sloppy\setboolean{publ}{false}}{\end{raggedright}\baselineskip20pt\sloppy}

%Publication style settings
%\newenvironment{bmcformat}{\fussy\setboolean{publ}{true}}{\fussy}



% Begin ...
\begin{document}
\begin{bmcformat}


%%%%%%%%%%%%%%%%%%%%%%%%%%%%%%%%%%%%%%%%%%%%%%
%%                                          %%
%% Enter the title of your article here     %%
%%                                          %%
%%%%%%%%%%%%%%%%%%%%%%%%%%%%%%%%%%%%%%%%%%%%%%

\title{The Blue Obelisk five years on}
 
%%%%%%%%%%%%%%%%%%%%%%%%%%%%%%%%%%%%%%%%%%%%%%
%%                                          %%
%% Enter the authors here                   %%
%%                                          %%
%% Ensure \and is entered between all but   %%
%% the last two authors. This will be       %%
%% replaced by a comma in the final article %%
%%                                          %%
%% Ensure there are no trailing spaces at   %% 
%% the ends of the lines                    %%     	
%%                                          %%
%%%%%%%%%%%%%%%%%%%%%%%%%%%%%%%%%%%%%%%%%%%%%%


\author{Charles A Darwin\correspondingauthor$^{1,2}$%
       \email{Charles A Darwin\correspondingauthor - charles@londonzoo.co.uk}%
      \and 
         Egon L Willighagen$^3$%
         \email{Egon L Willighagen - egon.willighagen@ki.se}
     \and 
         Rajarshi Guha$^4$%
         \email{Rajarshi Guha - guhar@mail.nih.gov}
     \and 
         Noel M O'Boyle$^5$%
         \email{Noel M O'Boyle - n.oboyle@ucc.ie}
      \and
         Jane E Doe\correspondingauthor$^2$%
         \email{Jane E Doe\correspondingauthor - jane.e.doe@cambridge.co.uk}
      }
      

%%%%%%%%%%%%%%%%%%%%%%%%%%%%%%%%%%%%%%%%%%%%%%
%%                                          %%
%% Enter the authors' addresses here        %%
%%                                          %%
%%%%%%%%%%%%%%%%%%%%%%%%%%%%%%%%%%%%%%%%%%%%%%

\address{%
    \iid(1)Life Sciences Department, Kings College London, Cornwall House,%
        Waterloo Road, London, UK\\
    \iid(2)Department of Zoology, Cambridge, Waterloo Road, London, UK\\
    \iid(3)Division of Molecular Toxicology, Institute of Environmental Medicine, %
        Nobels vaeg 13, Karolinska Institutet, 171 77 Stockholm, Sweden
    \iid(5)Analytical and Biological Chemistry Research Facility, Cavanagh Pharmacy Building, University College Cork, College Road, Cork, Co. Cork, Ireland
}%

\maketitle

%%%%%%%%%%%%%%%%%%%%%%%%%%%%%%%%%%%%%%%%%%%%%%
%%                                          %%
%% The Abstract begins here                 %%
%%                                          %%
%% The Section headings here are those for  %%
%% a Research article submitted to a        %%
%% BMC-Series journal.                      %%  
%%                                          %%
%% If your article is not of this type,     %%
%% then refer to the Instructions for       %%
%% authors on http://www.biomedcentral.com  %%
%% and change the section headings          %%
%% accordingly.                             %%   
%%                                          %%
%%%%%%%%%%%%%%%%%%%%%%%%%%%%%%%%%%%%%%%%%%%%%%


\begin{abstract}
        % Do not use inserted blank lines (ie \\) until main body of text.
        \paragraph*{Background:} Text for this section of the abstract. 
      
        \paragraph*{Results:} Text for this section of the abstract \ldots

        \paragraph*{Conclusions:} Text for this section of the abstract \ldots
\end{abstract}



\ifthenelse{\boolean{publ}}{\begin{multicols}{2}}{}




%%%%%%%%%%%%%%%%%%%%%%%%%%%%%%%%%%%%%%%%%%%%%%
%%                                          %%
%% The Main Body begins here                %%
%%                                          %%
%% The Section headings here are those for  %%
%% a Research article submitted to a        %%
%% BMC-Series journal.                      %%  
%%                                          %%
%% If your article is not of this type,     %%
%% then refer to the instructions for       %%
%% authors on:                              %%
%% http://www.biomedcentral.com/info/authors%%
%% and change the section headings          %%
%% accordingly.                             %% 
%%                                          %%
%% See the Results and Discussion section   %%
%% for details on how to create sub-sections%%
%%                                          %%
%% use \cite{...} to cite references        %%
%%  \cite{koon} and                         %%
%%  \cite{oreg,khar,zvai,xjon,schn,pond}    %%
%%  \nocite{smith,marg,hunn,advi,koha,mouse}%%
%%                                          %%
%%%%%%%%%%%%%%%%%%%%%%%%%%%%%%%%%%%%%%%%%%%%%%




%%%%%%%%%%%%%%%%
%% Background %%
%%
\section*{Background}
The Blue Obelisk group was established in 2005 at the
229\textsuperscript{th} National Meeting of the American Chemistry
Society as a response to the lack of open data, open standards and
open source (ODOSOS) in chemistry. While other scientific disciplines
such as physics, biology and astronomy (to name a few) were embracing
new ways of doing science and reaping the benefits of community
efforts, there was little if any innovation in the field of chemistry
and scientific progress was actively hampered by the lack of access to
data and tools.

The goals of the Blue Obelisk group were described in Guha et al.\cite{guha2006}. but are briefly summarised here...


%%%%%%%%%%%%%%%%%%%%%%%%%%%%
%% Main section %%
%%
\section*{Open Source}
  \subsection*{Progress}

Open Source toolkits for cheminformatics have now existed for nearly
ten years. During this period, some toolkits were developed form
scratch, whereas others were made Open Source by releasing in-house
codebases under liberal licenses. The primary toolkits currently in
active development are the Chemistry Development Kit, OpenBabel and
RDKit. These projects support a variety of languages natively (Java,
C++, Python), but by virtue of language wrappers, they can easily be
used from languages that they were not originally developed in. For
example, the CDK can be accessed from Python and OpenBabel can be
accesed from Java via SWIG wrappers. One important aspect of the
toolkits noted here is that they originated for the support of desktop
and server-side applications. 

More recently, a number of workers have developed light-weight
toolkits that are targeted towards web applications. One example of
this form is the Chemdoodle library from XXX. The focus of these
toolkits is to support client-side cheminformatics functionality,
allowing for rich, chemically-aware browser-based applications.

In terms of functionality, most Open Source libraries support commonly
ued cheminformatics features. At the same time, individual toolkits
have their own strenghts. For example, the CDK has an extensive set of
molecular descriptors useful for QSAR modeling applictions, but lacks
support for 3D structure generation. In contrast OpenBabel, supports
3D structure generation and optimization using a variety of
forcefields, but lacks a built in set of molecular descriptors. RDKit
on the other hand supports both features and in general is the most
comprehensive Open Source toolkit in terms of functionality. Table XXX
summarizes the features of these three toolkits.

\subsubsection*{Second-generation tools}

Although feature-rich and robust cheminformatics toolkits are useful
in and of themselves, they can also be seen as providing a base layer
on which additional tools and applications can be built. This is one of the reasons that cheminformatics toolkits are so important to the open source `ecosystem'; their availability lowers the barrier for the development of a `second generation' of chemistry software that no longer needs to concern itself with the low-level details of manipulating chemical structures, and can focus on providing additional functionality and ease-of-use.

Bioclipse (v1.0 released in Aug 2006 - ????within the scope of the paper???) and Open Babel (v1.0 in Oct 2009) are two examples of such software, based on the CDK and Open Babel, respectively. Bioclipse is an award-winning `molecular workbench' for life sciences that ..., while Avogadro is a 3D molecular editor and viewer aimed at preparing and analysis computational chemistry calculations. An interesting aspect of both of these projects is that they share some developers with the underlying toolkit and this has driven the development of new features in the CDK and Open Babel (??? is this true for the CDK too? ???).

Some other recent projects that build on Blue Obelisk software include AMBIT (a GUI that facilitates registration of chemicals for the REACH EU directive on toxicity, based on the CDK), etc.

\subsubsection*{The business end}

Although the software developed by the Blue Obelisk is Open Source on principle, this does not rule out the establishment of a business based on open source chemistry software. Rather than supporting itself through sales, such a business generally relies on providing support, customisation and services around the software. The interaction between an open source project and a company is important, and works best where there is some form of ``giving back" to the project (such as developer time, code contributions, or finanical assistance with server costs).

An example of such an interaction was the donation by eMolecules, Inc. to Open Babel of code for the canonicalisation of molecules and fragments (Nov 2006). eMolecules is an online vendor of chemicals that uses Open Babel under-the-hood to manage its compound collections. Another more recent example occurred in July 2010 when Silicos, a Belgian company that provides services
in the area of cheminformatics, released several command line applications based on Open Babel as well as donating code to the project. For example, the Pharao
tool released by them is a comprehensive solution for pharmacophore
searching that provides extensive support for a variety of
pharmacophore searches. Other tools released by them include Seive for
fitering by molecular property and XXX.

Another example of Open Source
tools originating from commercial groups is the ChemCraft tool from
Molecular Networks GmBH, that does XXX.

Rather than releasing new tools, an alternative approach to provide an interface to existing tools. hBar Solutions has developed an online portal, hBar Lab, for managing and performing computational chemistry calculations in the cloud. To do so it leverages two Blue Obelisk projects, Jmol and Open Babel, as well as the open source quantum mechanics packages MPQC.

  \subsubsection*{Collaboration and interoperability}

One of the effects of the Blue Obelisk has been to bring developers together from different Open Source chemistry projects so that they look for opportunities to collaborate rather than compete, and to leverage work done by other projects to avoid duplication of effort. As an example of this, when in March 2008 the Jmol development team were looking to add support for energy minimisation, rather than implement a forcefield from scratch they ported the UFF forcefield implementation from Open Babel to Jmol. This code has allowed Jmol to support 2D to 3D conversion of structures (through energy minimisation).

Another collaborative initiative between Blue Obelisk projects was the establishment in May 2008 of
the ChemiSQL project. This brought together the developers of several
open source chemistry database cartridges (PgChem, MyChem, OrChem and
more recently Bingo) with a view to making their database APIs more
similar and collaborating on benchmark datasets for assessing
performance. For two of these projects, PgChem and MyChem, which are both based on
Open Babel, there is the additional possibility of working together on a shared
codebase.

In the area of cheminformatics tookits, two of the existing toolkits Open Babel and RDKit are planning to work together on a common underlying framework called MolCore. This project is still in the planning stage, but if it is a success it will mean that the the two libraries will be interoperable (while retaining their existing focus) but also that the cost of maintaining the code will be shared among more developers, freeing time for the development of new features.

One of the goals of the Blue Obelisk is to promote interoperability in chemical informatics. When barriers exist to moving chemical data between different software, the community becomes fragmented and there is the danger of vendor lock-in (where users are constrained to using a particular software, a situation which puts them at a disadvantage). 

...cinfony...

  \subsection*{Remaining challenges}

Accuracy. Very often software work at the 95\% to 98\% level. The variety of chemical structures is such that with a large enough dataset, 'unusual' structures are always found which may be mishandled by software. Given that much of the development of open source software is unfunded, and relies heavily on developer motivation, it is understandable that working on the final N\% of problem structures (which may require substantial work) is not the most exciting of tasks. Putting a bounty system in place may be useful for these situations where a researcher needs a problem to be fixed because it affects a particular dataset of interest.

Performance. There is a famous quote by Knuth that "Premature optimization is root of all evil (or something)". Given that compute time is cheap, ....

Databases?

\section*{Open Standards}
  \subsection*{Progress}

The IUPAC InChI identifier is a non-proprietary and unique identifier for chemical substances designed to enable linking of diverse data compilations. Although its development predates the Blue Obelisk, software such as Open Babel has included InChI support since 2005. Since the official InChI implementation is in C, it is difficult to access from the other widely used language for cheminformatics toolkits, Java. The Blue Obelisk project JNI-InChI has been set up to solve this problem by using the Java Native Interface to link the InChI binary to Java. In this way, it promotes the wider adoption of this standard identifier by the chemistry community.

    \subsubsection*{OpenSMILES}

One of the most widely used ways to store chemical structures is the SMILES format (or SMILES string). This is a linear notation depicted by Daylight Information Systems that describes the connection table of a molecule and may optionally encode chirality. Its popularity stems from the fact that it is a compact representation of the chemical structure that is human readable and writable, and is convenient to manipulate (e.g. to include in spreadsheets, or copy from a Wikipedia article).

CML?

Pistoia Alliance (and others?) should be mentioned at this point

  \subsection*{Remaining challenges}

A core requirement for chemical structure databases and chemical
registration systems in general is the notion of structure
standardization.  That is,  for a given input structure, multiple
representations should be converted to one canonical form. 
Structure canonicalization routines partially address this aspect,
converting multiple alternative topologies to a single canonical
form. However, the problem of standardization is broader than just
topological canonicalization. Features that must be considered include
\begin{itemize}
\item topological canonicalization
\item handling of charges
\item tautomer enumeration and canonicalization
\item normalization of functional groups
\end{itemize}
Currently, most of the individual components of a ``standardization
pipeline'' can be implemented using BO tools. The larger problem is
that there is no agreed upon list of steps for a standardization
process. While some specifications have been published (e.g., Pubchem)
and some standardization services and tools are available (Pubchem
provides an online service to standaridize molecules and the NCGC
provides a stand alone tool) each group has their own set of rules. A
common reference specification for standardization would be of immense
value in interoperability between structure repositories as well as
between toolkits (though the latter is still confounded by differences
in lower level cheminformatic features such as aromaticity models).

We have already discussed the development of an Open SMILES standard. While much progress has been made towards a complete specification, more remains to be done before this can be considered finished. After that point, the next logical step would be to start work on a standard for the SMARTS language, the extension to SMILES that specifies patterns that match chemical substructures.

\section*{Open Data}
  \subsection*{Progress}
  \subsection*{Remaining challenges}



%%%%%%%%%%%%%%%%%%%%%%
\section*{Conclusions}

We have shown that the Blue Obelisk has been very successful
in bringing together researchers and developers with common interests
in ODOSOS, leading to development of many useful resources freely
available to the chemistry community. But how best to engage with the
wider chemistry community outside of the Blue Obelisk remains an open
question. If the Blue Obelisk is truly to make an impact,
then an attempt must be made to reach beyond the subscribers to the
BO mailing list and blogs of members.

One recent development
that attempts to address this issue was the establishment in Apr 2010 of a
question-and-answer
website related to Blue Obelisk projects and themes at
http://blueobelisk.shapado.com. This is a website in the
style of Stack Overflow that encourages high quality answers (and
questions) through the use of a voting system. In the year since it
was established, over
200 users have registered, many of whom had no previous involvement
with the Blue Obelisk.

We hope to see this involvement between the Blue Obelisk and the wider
community grow in future. To this end, we encourage the reader to
visit http://blueobelisk.org, send a message to our mailing list,
investigate related projects or read our blogs.


%%%%%%%%%%%%%%%%%%%%%%%%%%%%%%%%
\section*{Authors contributions}
   Charles Darwin did all the work. The others stole the glory. 
    

%%%%%%%%%%%%%%%%%%%%%%%%%%%
\section*{Acknowledgements}
  \ifthenelse{\boolean{publ}}{\small}{}
  Thanks to everyone.


 
%%%%%%%%%%%%%%%%%%%%%%%%%%%%%%%%%%%%%%%%%%%%%%%%%%%%%%%%%%%%%
%%                  The Bibliography                       %%
%%                                                         %%              
%%  Bmc_article.bst  will be used to                       %%
%%  create a .BBL file for submission, which includes      %%
%%  XML structured for BMC.                                %%
%%                                                         %%
%%                                                         %%
%%  Note that the displayed Bibliography will not          %% 
%%  necessarily be rendered by Latex exactly as specified  %%
%%  in the online Instructions for Authors.                %% 
%%                                                         %%
%%%%%%%%%%%%%%%%%%%%%%%%%%%%%%%%%%%%%%%%%%%%%%%%%%%%%%%%%%%%%


{\ifthenelse{\boolean{publ}}{\footnotesize}{\small}
 \bibliographystyle{bmc_article}  % Style BST file
  \bibliography{paper} }     % Bibliography file (usually '*.bib' ) 

%%%%%%%%%%%

\ifthenelse{\boolean{publ}}{\end{multicols}}{}

%%%%%%%%%%%%%%%%%%%%%%%%%%%%%%%%%%%
%%                               %%
%% Figures                       %%
%%                               %%
%% NB: this is for captions and  %%
%% Titles. All graphics must be  %%
%% submitted separately and NOT  %%
%% included in the Tex document  %%
%%                               %%
%%%%%%%%%%%%%%%%%%%%%%%%%%%%%%%%%%%

%%
%% Do not use \listoffigures as most will included as separate files

\section*{Figures}
  \subsection*{Figure 1 - Sample figure title}
      A short description of the figure content
      should go here.

  \subsection*{Figure 2 - Sample figure title}
      Figure legend text.



%%%%%%%%%%%%%%%%%%%%%%%%%%%%%%%%%%%
%%                               %%
%% Tables                        %%
%%                               %%
%%%%%%%%%%%%%%%%%%%%%%%%%%%%%%%%%%%

%% Use of \listoftables is discouraged.
%%
\section*{Tables}
  \subsection*{Table 1 - Blue Obelisk Open Source software projects}
    (Description if necessary XXXXXXXXXXXXXXX. Add citations to project names.)
    \par \mbox{}
    \par
    \mbox{
      \begin{tabular}{|c|c|c|}
        %% \hline \multicolumn{3}{|c|}{My Table}\\ \hline
        \hline Name & Website & Description? or Lead Developer? \\ \hline
        \multicolumn{3}{|c|}{Cheminformatics toolkits} \\ \hline
        Chemistry Development Kit & http://cdk.sf.net & XXXX  \\ \hline
        Cinfony & http://cinfony.googlecode.com & Noel O'Boyle, Python interface to toolkits \\ \hline
        Indigo & http://ggasoftware.com/opensource/indigo & GGA Software \\ \hline
        Open Babel & http://openbabel.org & Geoffrey Hutchison et al \\ \hline
        RDKit & http://rdkit.org & Greg Landrum \\ \hline
        \multicolumn{3}{|c|}{Interconversion} \\ \hline
        OSRA & http://osra.sf.net & Igor Filippov, Image to structure \\ \hline
        A3 & ..  & .  \\ \hline
      \end{tabular}
      }
  \subsection*{Table 2 - Sample table title}
    Large tables are attached as separate files but should
    still be described here.



%%%%%%%%%%%%%%%%%%%%%%%%%%%%%%%%%%%
%%                               %%
%% Additional Files              %%
%%                               %%
%%%%%%%%%%%%%%%%%%%%%%%%%%%%%%%%%%%

\section*{Additional Files}
  \subsection*{Additional file 1 --- Sample additional file title}
    Additional file descriptions text (including details of how to
    view the file, if it is in a non-standard format or the file extension).  This might
    refer to a multi-page table or a figure.

  \subsection*{Additional file 2 --- Sample additional file title}
    Additional file descriptions text.


\end{bmcformat}
\end{document}







